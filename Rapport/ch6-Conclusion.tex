\chapter{Conclusion}
\thispagestyle{fancy}

Nous retiendrons que la société \textsc{Roturo}, fondée par un ancien employé de la société FIRM est fautive sur plusieurs points. Monsieur \textsc{Roturo} non seulement viole la clause de non-concurrence qu’il a signé dans son contrat le liant à la société \textsc{Firm} de par la création de sa société dans le même secteur d’activité avant la fin de cette clause, mais essaye aussi de tirer profit de la notoriété de son ancien employeur et de son image, se rendant ainsi coupable de parasitisme. En restant dans le secteur de l’information, la société \textsc{Roturo} utilise des pratiques commerciales trompeuses pour améliorer son image, illustrées notamment par la déclaration portant sur le fait que son produit est le seul sur le marché ayant un certain composant dans tous les constituants de son bracelet, fait invalidé au cours de cet étude. On pourrait aussi citer le démarchage illégal des clients de son ancien employeur, contraire à une clause présente dans son contrat de travail.\\
Cet étude a aussi permis de mettre en lumière le non-respect de la propriété intellectuelle par la société \textsc{Roturo}, que ce soit au niveau de la propriété industrielle, illustré par la violation du brevet déposé par la société \textsc{Firm} le 5 mai 2014, ou de la propriété littéraire et artistique, mis en lumière par la transgression du droit d’auteur sur le site internet, ainsi que sur le logo de l’entreprise.\\
Enfin, en appliquant les recommandations présentes à la fin de notre étude, l’entreprise \textsc{Firm} pourrait anticiper ou même éviter de nouveaux déboires avec la justice, améliorer la communication interne et externe, mais aussi affaiblir la société concurrente \textsc{Roturo}.
