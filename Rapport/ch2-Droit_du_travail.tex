\chapter{Droit du travail}
\thispagestyle{fancy} 

\section{Qu'est-ce ?}
\emph{   Le droit du travail régit l’ensemble des rapports juridiques qui naissent du contrat de travail subordonné ou dépendant. Il ne régit que le travail pour le compte d’autrui. Il ne concerne donc pas le travail de celui qui œuvre pour son propre compte (travailleur indépendant). Le droit du travail ne saisit pas toutes les formes d’activité professionnelle, mais uniquement le travail subordonné.}(\textsc{http://www.cours-de-droit.net}) \\\\
Dans cette étude le droit du travail consiste en l'analyse du contrat de travail pour en révéler les forces et les faiblesses.\\\\
Pour ce faire nous nous appuierons sur les diverses notes en notre possession, les codes (en particulier celui du travail) et les rapports d'audiences publiques de jurisprudence.

\section{Rapport d'étude du cas de la \emph{sarl firm}}
\subsection{Validité de la clause de non-concurrence}
\subsubsection{Question de droit}
La clause de non-concurrence est-elle valide ?
\subsubsection{Règles juridiques}
Selon l'\textit{Accord du 17 avril 2008 relatif à la clause de non-concurrence section 1 sous section 1}.\\
La clause de non-concurrence, pour être licite, est soumise à des conditions de fond cumulatives et de forme que doivent respecter les parties au contrat de travail.
Ainsi, elle doit :
\begin{itemize}
\item[-] être indispensable aux intérêts légitimes de l'entreprise rapportés aux spécificités de l'emploi du salarié ;
\item[-] être limitée quant à son application dans le temps et l'espace ;
\item[-] et comporter au bénéfice du salarié le versement d'une contrepartie financière proportionnée aux atteintes que cette clause porte à la liberté de travail de celui-ci. Ce versement intervient postérieurement à la rupture du contrat de travail.
\end{itemize}
\subsubsection{Argumentations}
\begin{itemize}
\item[-] L'employé étant directeur des systèmes d'information, il a accès à l'ensemble des informations de l'entreprise. Il faut donc éviter l'utilisation de telles informations que se soit à l'encontre de l'entreprise ou pour un concurrent.
\item[-] Selon le contrat de travail, \emph{Le délai d'un an commencera à courir le jour où Monsieur Christian ROTURO cessera effectivement de faire partie du personnel de la Société, à savoir à la fin de son préavis.}. La durée de la clause est définie à un an. Une autre durée est cité plus haut mais celle-ci étant plus importante (trois ans) et donc en défaveur de l'employé, la durée d'un an est retenue. De plus \emph{cet engagement de non-concurrence de Monsieur Christian \textsc{Roturo} couvre la France entière}, l'application est limité dans l'espace.
\item[-] Selon le contrat de travail, \emph{en cas d’activation de la clause de non-concurrence, une indemnité de 1000 \euro par mois sera versée à Monsieur Christian \textsc{Roturo}}. La rémunération a été définie à 1000 \euro par mois pour cette clause.
\end{itemize}
\subsubsection{Conclusion}
La clause de non-concurrence remplit les conditions définies par la loi. Elle est donc valide.

\subsection{Respect de la Non-concurrence}
\subsubsection{Question de droit}
Après avoir été licencié, est-il légal de fonder une société dans le même secteur d'activité que l'entreprise que l'employé vient de quitter ?
\subsubsection{Règles juridiques}
Selon le \textit{Bulletin 1997 V N 460 p. 327 de la  Cour d'appel de Rennes , du 14 mars 1995}
Clause de non-concurrence - Violation - Emploi dans une entreprise ayant une activité concurrente - Nature de l'emploi - Recherche nécessaire .\\
Pour déterminer si un salarié a violé une clause de non-concurrence qui lui interdisait toute activité portant, sous une forme quelconque, sur la promotion et la commercialisation de produits susceptibles de concurrencer les produits sur lesquels aura porté son activité de chef de vente régional, une cour d'appel doit rechercher quelle est la nature de l'activité de ce salarié dans l'entreprise concurrente et ne peut se borner à constater qu'il n'est pas établi qu'il ait participé à une vente de produits identiques à ceux de l'entreprise bénéficiaire de la clause de non-concurrence. 
\subsubsection{Argumentations}
Dans le contrat de travail, il est stipulé que "\emph{[Il est interdit] [...] d'entrer au service d'une entreprise exerçant une activité similaire à celle de la Société ou pouvant la concurrencer ; de s'intéresser directement ou indirectement, sous quelque forme que ce soit, à une entreprise ayant une activité similaire à celle de la Société. [...] Le délai d'un an commencera à courir le jour où Monsieur Christian \textsc{Roturo} cessera effectivement de faire partie du personnel de la Société, à savoir à la fin de son préavis.}"\\
La \textsc{sas Roturo} comme la \textsc{sarl Firm} sont toutes deux dans le meme domaine d'activité. Ces deux sociétés possèdent le meme \emph{code NAF (2620Z : Fabrication d'ordinateurs et d'équipements périphériques)}.\\
De plus sa société a été fondé le 5 juin 2011, or il a été licencié en décembre 2010 : cela entre en désaccord avec le délai d'un an évoqué dans le contrat de travail section 13.
\subsubsection{Conclusion}
Le dit Christian \textsc{Roturo} étant au sein d'une entreprise de la même catégorie que celle de son ex-employeur, dans un délai inférieur à celui stipulé dans son contrat de travail viole la clause de non-concurrence. Celle-ci n'étant pas respecté, M. \textit{Roturo} devra verser à son employeur une indemnité forfaitaire fixée à la somme de 4 000 \euro. Nous demanderons aussi la fermeture de l'entreprise de M. \textit{Roturo} de par sa création illégale à l'époque.

\subsection{Le démarchage de clients de l'ex-employeur}
\subsubsection{Question de droit}
Le Démarchage, direct ou indirect, des clients de son ancien employeur constitue-t-il un manquement aux obligations du contrat de travail ?
\subsubsection{Règle juridique}
\textit{Audience publique du mercredi 8 janvier 1997, N de pourvoi: 93-44009:}
Et attendu, ensuite, qu'ayant rappelé qu'aux termes de l'article 41 de la convention collective applicable, tout salarié quittant, pour quelque cause que ce soit, un employeur relevant de ladite convention, s'interdit formellement de démarcher, directement ou indirectement, la clientèle appartenant à l'employeur qu'il vient de quitter, la cour d'appel, devant laquelle il n'était pas soutenu par la salariée qu'elle n'avait pas connaissance de la convention collective applicable et qui a constaté que la salariée avait accompli aussitôt après la rupture du contrat de travail des actes positifs de démarchage auprès de clients de son ancien employeur, a, sans encourir les griefs du moyen, légalement justifié sa décision ; PAR CES MOTIFS : REJETTE le pourvoi.
\subsubsection{Argumentation}
Le \textit{contrat de travail signé par M \textsc{Roturo} avec la société \textsc{Firm}} stipule, parmi les obligations de l'employé:
\emph{"9. OBLIGATIONS [...] d/ Tout client acquis par Monsieur Christian \textsc{Roturo} le sera au nom de la Société et le restera après le terme du présent contrat."}
La société de M \textsc{Roturo} ayant démarché des clients de la société \textsc{Firm} (\textsc{sas Darty} et \textsc{sa Fnac}), ce dernier a donc manqué à cette obligation, et ce, même s'il n'a pas effectué le démarchage lui-même:
en effet, La société De M \textsc{Roturo} portant son nom, il est clair qu'il a, au moins indirectement, démarché les clients de son ancien employeur.
\subsubsection{Conclusion}
M \textsc{Roturo} a manqué à ses obligations en tant qu'ex-employé en démarchant les clients de la société \textsc{Firm}. En temps que tel, M \textsc{Roturo} devra verser l'équivalent de la moitié du bénéfice de la \textsc{sas Roturo} depuis le démarchage des clients de la \textsc{sarl Firm}. Nous demandons aussi la cessation de toute relation avec ceux-ci.

