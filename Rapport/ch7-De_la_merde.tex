citation dans l'intro \emph{•}
nomination \textsc{•}

nom de famille et nom de firm \textsc{}

titre d'article \textit{•}
citation dans l'argumentation \emph{•}


hugo : conseil
alexis : woft
moi : list

demande reconventionnel
dommage et interet
Roturo viol la symenteque

en reponse a : "la société ne l'engage pas elle a été creer par" dans le jurisprudence lf "non creer par soi"

\section{Note en vrac}

dans le cas ou l'employé aurait eu accès à des données ou des innovations techniques et/ou technologiques, il aurait du d'une part en informer la hiérarchie, d'autre part effectuer les démarches nécessaires au dépôt des brevets des innovations (\textit{Contrat de travail, section 11, paragraphe 2}).


\paragraph{pratiques commerciales douteuses }

Anciennement Concurrence déloyal (40 pour cent moins cher dans les memes magasins avec le même site, intention de nuire pendant les soldes :P) Directive n 2006/114/CE du 12 décembre 2006 du parlement européen et du conseil de l’union européeen en matière de publicité trompeuse et de publicité compararative

---------------------------------------------------------------------------------------
---------------------------------------------------------------------------------------
on chiffre le copie colle des mentions légales du sites à combien >< +mot clé
note en vrac : site cope colle => design du site copie colle
précision sur les keywords du site de roturo : parasitage par nom de société et nom de produit (firm watch))
---------------------------------------------------------------------------------------
---------------------------------------------------------------------------------------
---------------------------------------------------------------------------------------
---------------------------------------------------------------------------------------
---------------------------------------------------------------------------------------\\
Article 700 de la procedure civil
Le juge condamne la partie tenue aux dépens ou qui perd son procès à payer :
1 A l'autre partie la somme qu'il détermine, au titre des frais exposés et non compris dans les
dépens ;
2 Et, le cas échéant, à l'avocat du bénéficiaire de l'aide juridictionnelle partielle ou totale une
somme au titre des honoraires et frais, non compris dans les dépens, que le bénéficiaire de l'aide
aurait exposés s'il n'avait pas eu cette aide. Dans ce cas, il est procédé comme il est dit aux alinéas 3
et 4 de l'article 37 de la loi n 91-647 du 10 juillet 1991.
Dans tous les cas, le juge tient compte de l'équité ou de la situation économique de la partie
condamnée. Il peut, même d'office, pour des raisons tirées des mêmes considérations, dire qu'il n'y a
pas lieu à ces condamnations. Néanmoins, s'il alloue une somme au titre du 2 du présent article,
celle-ci ne peut être inférieure à la part contributive de l'Etat.
---------------------------------------------------------------------------------------\\
Article L335-2 de code propriété intel
Toute édition d'écrits, de composition musicale, de dessin, de peinture ou de toute autre production,
imprimée ou gravée en entier ou en partie, au mépris des lois et règlements relatifs à la propriété des
auteurs, est une contrefaçon et toute contrefaçon est un délit.
La contrefaçon en France d'ouvrages publiés en France ou à l'étranger est punie de trois ans
d'emprisonnement et de 300 000 euros d'amende.


******************************************************************************************
******************************************************************************************
******************************************************************************************

***** Faute : *****

***** Contexte :*****
La société Roturo a essayé d'entretenir la confusion en imitant illégalement le produit commercialisé par SARL
FIRM (FIRM détient le brevet sur le produit) , en commercialisant son produit dans les mêmes lieux de
distributions que FIRM , en créant un sigle proche de celui de la société concurrente ,en abaissant le prix de son
produit à -40 % tout en s'adonnant à une activité de concurrencé déloyale .
Règles juridiques :

***** Préjudice : *****

***** Faits : *****
Il y a un préjudice Futur , un préjudice réel et un préjudice certain.
----- 1.Un préjudice futur : -----
En détournant nos clients tout en s'adonnant à une concurrence déloyale , ROTURO est responsable d'une perte
future évaluée à -60 % sur notre chiffre d'affaire . Seul le juge peut statuer de la validité de ce préjudice .
-----2.Un préjudice réel :-----
FIRM a investit une somme importante dans sa compagne publicitaire estimée à 175000 euros , le fait de
détourner les clients de FIRM , ROTURO est responsable de l'échec de cette compagne qui avait pour but
d'attirer et non d'éloigner des clients .
-----3.Un préjudice certain:-----
Est le même que celui du préjudice réel du fait que l'effet de celui-ci est toujours en cours (la société continue à
perdre au profit de ROTURO au moment où on parle) .

*****Règles juridiques :*****
Un préjudice futur, est un préjudice qui n'est pas encore survenu au moment où le juge statue, peut d'ores et déjà
donner lieu à indemnisation dès lors que sa survenance future est certaine. Le préjudice réel : Il existe un lien de
causalité entre l'acte de la personne désignée comme responsable et le préjudice est direct et la concerne
personnellement . Le préjudice est certain lorsqu'il est établi au moment où il est invoqué même si son effet n'est
pas immédiat . Un préjudice futur peut en effet être indemnisé s'il est certain qu'il se produira et s'il peut être
évalué immédiatement .




Dommages et intérêts :

Faits:
La société SARL FIRM demande 3 millions d'euros pour dommages et intérêts conformément au premier
paragraphe de l'article 700 de la procédure civil . Le juge tranchera sur la somme finale dans le cas où la société
ROTULO sera condamnée . Le juge déterminera aussi si la somme sera versé à la societé FIRM en une fois sous
la forme d'un capital , ou sous forme d'une rente qui peut être indexée .

Règles juridiques (lois) :
Article 700 de la procédure civil
Le juge condamne la partie tenue aux dépens ou qui perd son procès à payer :
1 A l'autre partie la somme qu'il détermine, au titre des frais exposés et non compris dans les
dépens ;
2 Et, le cas échéant, à l'avocat du bénéficiaire de l'aide juridictionnelle partielle ou totale unesomme au titre des honoraires et frais, non compris dans les dépens, que le bénéficiaire de l'aide
aurait exposés s'il n'avait pas eu cette aide. Dans ce cas, il est procédé comme il est dit aux alinéas 3
et 4 de l'article 37 de la loi n 91-647 du 10 juillet 1991.
Dans tous les cas, le juge tient compte de l'équité ou de la situation économique de la partie
condamnée. Il peut, même d'office, pour des raisons tirées des mêmes considérations, dire qu'il n'y a pas lieu à ces condamnations. Néanmoins, s'il alloue une somme au titre du 2 du présent article,
celle-ci ne peut être inférieure à la part contributive de l'Etat.