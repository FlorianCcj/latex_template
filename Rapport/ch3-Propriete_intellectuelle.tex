\chapter{Propriété intellectuelle}
\thispagestyle{fancy} 

\section{Qu'est-ce ?}

\emph{Le terme “propriété intellectuelle” désigne les œuvres de l’esprit : inventions; œuvres littéraires et artistiques; dessins et modèles; et emblèmes, noms et images utilisés dans le commerce.}\textsc{(Organisme Mondial de la Propriété Intellectuelle)}

\section{Rapport d'étude du cas de la \textsc{sarl Firm}}

\subsection{Utilisation de brevet}
\subsubsection{Question de droit}
Une société a-t-elle le droit d'utiliser le concept breveté d'une autre société ?
\subsubsection{Règles juridiques}
\textit{Article L613-3 du code de propriété intellectuelle : }\\
Sont interdites, à défaut de consentement du propriétaire du brevet :
\begin{itemize}
\item[a] La fabrication, l'offre, la mise dans le commerce, l'utilisation, l'importation, l'exportation, le transbordement, ou la détention aux fins précitées du produit objet du brevet ;
\item[b] L'utilisation d'un procédé objet du brevet ou, lorsque le tiers sait ou lorsque les circonstances rendent évident que l'utilisation du procédé est interdite sans le consentement du propriétaire du brevet, l'offre de son utilisation sur le territoire français ;
\item[c] L'offre, la mise dans le commerce, l'utilisation, l'importation, l'exportation, le transbordement ou la détention aux fins précitées du produit obtenu directement par le procédé objet du brevet.
\end{itemize}
\textit{Article L511-9 du code de propriété intellectuelle.}\\
La protection du dessin ou modèle conférée par les dispositions du présent livre s'acquiert par l'enregistrement. Elle est accordée au créateur ou à son ayant cause.\\
L'auteur de la demande d'enregistrement est, sauf preuve contraire, regardé comme le bénéficiaire de cette protection.
\subsubsection{Argumentation}
La montre connectée de la société \textsc{Roturo} utilise une technologie de montre modulable dans la mesure où le cadran peut être inter-changé. \\
La montre connecté \textsc{Roturo} utilise, comme la montre modulaire \textsc{Firm}, faisant l’objet du brevet, des connections jack 3.5 mm.\\
La montre de la société \textsc{Roturo} reprend l'apparence générale de la montre de la société \textsc{Firm} pourtant déposée en tant que dessins et modèles :\\
\begin{itemize}
\item Les maillons aux formes arrondies sont reliés les uns aux autres par le milieu de leur côté, sans pièce intermédiaire, un maillon s'enfichant perpendiculairement à son voisin.\\
\item Le cadran, dans sa version à affichage numérique, se confond dans les autres maillons en se 
distinguant uniquement par une longueur supérieure et la présence d’un affichage numérique.
\end{itemize}
Or ce modèle de design a été déposé par la \textsc{sarl Firm} le 5 mai 2014, ainsi que leur brevet.
\subsubsection{Conclusion}
La montre connectée commercialisée par \textsc{Roturo} constitue une violation du brevet/modèle déposé pour la montre modulaire \textsc{Firm}.\\
La société ROTURO a profité des efforts en recherche et développement de FIRM pour commercialiser une contrefaçon, causant de ce fait un préjudice, estimé à 1 156 389 euros  et 21 centimes de chiffre d'affaires. En outre, en commercialisant une contrefaçon de sa montre « connected watch 2.0 », ROTURO expose FIRM à un préjudice futur certain sur son chiffre d'affaires, estimé à 1 602 541 euros et 48 centimes. 

\section{Demande reconventionnel}

\subsection{Désinformation}
\subsubsection{Question de droit}
Peut-on indiquer de fausses informations dans le but d'augmenter ses profits ?
\subsubsection{Règles juridiques}
\textit{Article L121-1 du code de la consommation}\\
I.-Une pratique commerciale est trompeuse si elle est commise dans l'une des circonstances suivantes :\\
\begin{enumerate}
\item Lorsqu'elle crée une confusion avec un autre bien ou service, une marque, un nom commercial, ou un autre signe distinctif d'un concurrent ;\\
\item Lorsqu'elle repose sur des allégations, indications ou présentations fausses ou de nature à induire en erreur et portant sur l'un ou plusieurs des éléments suivants :\\
\begin{enumerate}
\item L'existence, la disponibilité ou la nature du bien ou du service ;\\
\item Les caractéristiques essentielles du bien ou du service, à savoir : ses qualités substantielles, sa composition, ses accessoires, son origine, sa quantité, son mode et sa date de fabrication, les conditions de son utilisation et son aptitude à l'usage, ses propriétés et les résultats attendus de son utilisation, ainsi que les résultats et les principales caractéristiques des tests et contrôles effectués sur le bien ou le service ;\\
\item Le prix ou le mode de calcul du prix, le caractère promotionnel du prix et les conditions de vente, de paiement et de livraison du bien ou du service ;\\
\item Le service après-vente, la nécessité d'un service, d'une pièce détachée, d'un remplacement ou d'une réparation ;\\
\item La portée des engagements de l'annonceur, la nature, le procédé ou le motif de la vente ou de la prestation de services ;\\
\item L'identité, les qualités, les aptitudes et les droits du professionnel ;\\
\item Le traitement des réclamations et les droits du consommateur ;\\
\end{enumerate}
\item Lorsque la personne pour le compte de laquelle elle est mise en oeuvre n'est pas clairement identifiable. 
\end{enumerate}
\textit{Article L213-1 du code de la consommation}\\
Sera puni d'un emprisonnement de deux ans au plus et d'une amende de 300 000 euros quiconque, qu'il soit ou non partie au contrat, aura trompé ou tenté de tromper le contractant, par quelque moyen ou procédé que ce soit, même par l'intermédiaire d'un tiers\\
\subsubsection{Argumentations}
Dans le 22 minutes du Vendredi 8 Janvier 2016, la publicité pour la \emph{Connected Clock 2.0 par \textsc{Roturo}} précise que leur invention est \emph{la seule montre connectée équipée de capteurs de mouvement dans tout les maillons du bracelet}. Or la montre \textsc{Firm} déposé deux ans auparavant (juin 2014) peut grâce, à ses maillons interchangeables \emph{(motion ou gesture control)} n'avoir que des maillons de détection de mouvement. \\
Les allégations de la \textsc{sas Roturo} sont trompeuses sur leur service, il s'agit donc d'une  pratique commerciale trompeuse.
\subsubsection{Conclusion}
La \textsc{sas Roturo} a donc recours à des pratiques commerciales trompeuses, pratiques illégales, dans le but d'augmenter ses profits. De ce fait, le ou les responsables sont passibles de deux dans d'emprisonnement, ainsi que d'une amende de 300 000 euros.
