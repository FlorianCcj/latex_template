\chapter{Droit des nouvelles technologies}
\thispagestyle{fancy} 

\section{Qu'est-ce ?}

\emph{Internet, e-mail, SMS, 4G, chat, smartphones, tablettes, réseaux sociaux... sont des notions encore récentes et regroupées sous l'appelation des Technologies de l'Information et des Communications. Cette rubrique traite de l'ensemble des questions juridiques liées à l'informatique, au Droit de l'internet, aux noms de domaine, mais aussi celles relatives au développement de la bureautique et des téléservices, de la téléphonie, des réseaux sociaux, du e-commerce, de la biométrie, etc.\\
La protection des données personnelles, la responsabilité des acteurs et intermédiaires, la signature électronique, le paiement sans contact et plus largement le droit des contrats dématérialisés, font l'objet d'évolutions juridiques constantes, sans oublier la pratique des liens commerciaux ou encore le cybersquattage.} (\textsc{http://www.net-iris.fr/veille-juridique/droit-technologies/})

\section{Rapport d'étude du cas de la \emph{sarl Firm}}

\subsection{Parasitisme}
\subsubsection{Question de droit}
La société \textsc{Roturo} a-t-elle tenté de jouir de la notoriété de la société \textsc{Firm} sur internet ?
\subsubsection{Règles juridiques}
Dans le cadre d'une action pour parasitisme, \textit{l'article 1382 du code civil} s'applique :
Tout quelconque de l'homme, qui cause à autrui un dommage, oblige celui par la faute duquel il est arrivé à le réparer.
Il est nécessaire d'apporter la preuve des éléments suivant : la faute, le dommage et le lien de causalité
\subsubsection{Argumentations}
Pour justifier la faute de la société \textsc{Roturo}, on peut tout d'abord citer la présence dans les mots clés définissant la page d'accueil de ladite société des termes suivants : \emph{"firm"} et \emph{"firm watch"}. Ces mots sont utilisés par les sites de recherches comme \textsc{Google} pour afficher les pages les plus pertinentes par rapport aux recherches de l'utilisateur.\\
Dans le même domaine de la recherche, la société \textsc{Roturo} s'est placée sur les mots clefs \emph{"connected watching"} et \emph{"connected watch"}. On remarque que le premier mot clef ressemble fortement au nom du produit de la société \textsc{Firm} \textit{"connected watch"}, et que le second est exactement le même.
Enfin, on peut aussi observer la ressemblance frappante entre les noms de domaines achetés par la société \textsc{Roturo} qui sont \emph{"connectedwatching.com"}, \emph{"connectedwatching.net"} et \emph{"connectedclockwatch.fr"}, avec le nom du produit de la société \textsc{Firm} \emph{"connected watch"}.\\
Ainsi, de par le fonctionnement du référencement passif s'effectuant notamment sur le nom du site, ses mots clefs et la popularité du site, il est évident que les mots clefs et noms de domaine mentionnés permettent d'augmenter la notoriété de la société \textsc{Roturo} sur internet. De plus, en se plaçant sur les mots clefs précédemment cités pour \textsc{Google Ads}, dans le domaine du référencement actif, le site ainsi référencé permet de profiter de la notoriété du produit et donc de la société \textsc{Firm}.\\
A cause de ces éléments, lorsqu'un client potentiel recherche le produit de la société \textsc{Firm}, celui-ci retrouve non seulement le site de la société \textsc{Roturo} avant le site de la société \textsc{Firm} dans le sens de la hauteur, mais aussi juste après. Le site de la société \textsc{Roturo} apparait donc lors de la recherche spécifique sur le produit de la société \textsc{Firm} sur le site \textit{google.fr}. \\
\subsubsection{Conclusion}
La société \textsc{Roturo} utilise donc la notoriété de la société \textsc{Firm} pour sa propre promotion sans l'accord de ladite société.\\
FIRM a investi une somme importante dans sa campagne publicitaire estimée à 600 000 euros. En détournant les clients de FIRM, ROTURO profite indument de cette campagne qui avait pour but d'attirer les acheteurs potentiels vers ses produits. La société FIRM a également dépensé une somme estimée à 75 000 euros pour la création de son site internet, repris à l'identique par ROTURO, qui profite ainsi de cette somme pour son propre bénéfice. La somme de 563 286 euros et 92 centimes est donc réclamée à ROTURO pour couvrir le préjudice subi.