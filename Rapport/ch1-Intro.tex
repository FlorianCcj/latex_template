\chapter*{Introduction}
\thispagestyle{fancy} 
\section{But du document}
Ce document présente l'étude du cas \textsc{sarl Firm} en opposition à \textsc{sas Roturo} réalisé dans le cadre du Challenge économique et juridique des entreprises au sein de l'\textsc{ESME Sudria}. 
\section{Contexte}
\subsection{la \textsc{sarl Firm}}
\textsc{Firm SARL} a été fondée par Adrien \textsc{mirf}, diplômé de l'\textsc{ESME} en 2004, et Georges \textsc{Lejarret}, diplômé de l'\textsc{Essec}. \\
Elle a été déposée par Adrien \textsc{Mirf} en 2007 à l'INPI dans la classe 9 en reprenant tous les mots de cette classe (tels que définis dans la classification de Nice). \\
\textsc{Firm} a un CA de 26 millions d'Euros en 2013 et 2014 et emploie 26 personnes. Son siège se situe au 5 rue de Rennes à Paris. \\ 
Ayant d'abord œuvré en tant que bureau d'études pour des entreprises souhaitant ajouter une dimension "connectée" à leurs produits, notamment dans les secteurs de l'automobile, de l'électroménager et du jouet, elle a ensuite décidé de commercialiser ses propres produits, profitant ainsi de son expérience dans le domaine. \\
Ils développent puis brevettent en mai 2014 une montre connectée modulaire commercialisé sous le nom de \textit{"connected watch"}. Ils déposent également un modèle de son design. Ils la mettent sur le marché en juin de la même année. La société \textsc{Firm} cède, par acte sous seing privé uniquement, une licence de fabrication de son invention à la société \textsc{Clock}. \\
Afin de mieux faire connaitre son produit, Adrien \textsc{Mirf} effectue une campagne publicitaire (évaluée à 600 000 \euro) un mois plus tard. Un site internet (évalué à 75 000 \euro) est également crée en 2013. \\\\
\subsection{Le début de la \textsc{sas Roturo}}
Christian \textsc{Roturo} est embauché chez \textsc{Firm sarl} le 27 septembre 2007 en tant que directeur des systèmes d'information, pour une rémunération annuelle brute de 65 000 \euro. Le contrat comprend une clause de non-concurrence. Il est licencié pour faute grave en décembre 2010 pour avoir utilisé son ordinateur professionnel à des fins personnelles, violant ainsi la charte informatique de l'entreprise, qu'il a accepté en vertu de la section 2 de celle-ci.\\
Il créé ensuite la société \textsc{Roturo} le 5 juin 2011 dans le but de fabriquer et commercialiser des objets connectés. \\
Son siège se situe au 15, place de la Défense. Il dépose le 5 août 2014 une marque complexe nationale dans la classe 9. Il reprend uniquement le mot "informatique" de la classe (tels que définis dans la classification de Nice). Le logo associé à la marque présente des similitudes avec celui de \textsc{Firm}, bien que ce dernier n'ait jamais été déposé. \\
\subsection{Le commencement d'un conflit}
En septembre 2014, il commence à commercialiser une montre connectée à cadran interchangeable sous le nom de \textit{"connected clock"}. La firme \textsc{Roturo} acquiert les noms de domaines \textit{"connectedwatching.com"}, \textit{"connectingwatch.net"} et \textit{"connectedclockwatch.fr"}, redirigeant tous vers un site internet publicitaire ressemblant fortement à celui de \textsc{Firm}. Il souscrit également au service \textit{"Google adwords"} et se place sur les termes \textit{"connected watching"} et \textit{"connected watch"}. Les termes \textit{"firm"} et \textit{"firm watch"} sont renseignés comme mots-clefs du site internet de \textsc{Roturo}.\\
\subsection{L'action de la justice}
Adrien \textsc{Mirf} estime que cette montre ressemble fortement à la \textit{"connected watch"} de son entreprise, et constate que les deux produits sont commercialisés dans les mêmes magasins (\textsc{sas Darty} et \textsc{sa Fnac}), et que la montre de \textsc{Roturo} affiche un prix 40\% inférieur à celui de la sienne.\\ 
La société \textsc{Firm} décide fin 2014 d'agir en justice contre la société \textsc{Roturo}. Le 2 octobre 2014, la société \textsc{Firm} fait procéder, dans la \textsc{sas Darty}, à l'achat, par un huissier, d'un exemplaire du produit de la société \textsc{Roturo}. Elle fait analyser dans son laboratoire de recherche ledit produit. La société \textsc{Firm} fait également procéder à une saisie contrefaçon dans les locaux de la société \textsc{Roturo}. \\
Cinq semaines après cette saisie, La société \textsc{Firm} assigne au tribunal de commerce de Paris la société \textsc{Roturo}. Malheureusemnt, après deux mois, le tribunal de commerce s'estime incompétent au motif que le litige porte sur de la contrefaçon et renvoie l'affaire devant le tribunal de grande instance de Paris.\\
\subsection{L'appel à l'étude}
En décembre 2015, la société \textsc{Firm} lance la phase 2 de sa montre modulaire: la \textit{"connected watch 2.0"}. Des capteurs de mouvement sont ajoutés dans le cadran et la montre communique avec une application mobile. La commercialisation du produit est prévue pour mars 2016. Aucun dépôt n'est effectué à l'\textsc{Inpi} afin de garder son projet secret.
En janvier 2016, la société \textsc{Roturo} lance la commercialisation de la \textit{"connected clock 2.0"}, équipée de capteurs de mouvements dans ses maillons. Aucun dépôt en propriété intellectuelle n'est déposé concernant ce nouveau produit. \\
Adrien \textsc{Mirf} estime que cette montre emprunte les caractéristiques de son propre produit, alors en attente de commercialisation.\\
La société \textsc{Firm} décide, à nouveau, d'attaquer en justice la société \textsc{Roturo} avec le soutien de la société \textsc{Clock}, cette fois au sujet du produit \textit{"connected clock 2.0"}.
\section{Objectif}
L'objectif de cette étude est de défendre les intérêts de la \textsc{sarl Firm} et de l'accompagner dans son action en justice intentée à l'encontre de la \textsc{sas Roturo}. Nous apporterons aussi quelques éléments permettant de renforcer les divers défenses de la société \textsc{Firm} ainsi que quelques angles d'attaque d'un point de vue stratégique.