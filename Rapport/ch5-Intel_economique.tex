\chapter{Intelligence economique}
\thispagestyle{fancy}

\section{Liste des dysfonctionnements}
Dans l'articles du journal \textit{Les Echos du 12 décembre 2014}, le dirigeant de la \textsc{sarl Firm} donne de trop nombreuses informations sur l'entreprise.\\
\begin{itemize}
\item \emph{"C’est Georges [Lejarret, note de la rédaction] qui est chargé des finances et de l’administration. Il s’occupe également des relations avec les sous-traitants, comme la société Clock, qui fabrique nos montres, avec l’aide et les conseils de notre avocat, Maître Dugenou."}
\item \emph{"A.M. : Nous nous partageons le travail entre associés. Pour notre activité d’équimentier automobile, c’est Georges, les jouets, c’est moi ; et pour les montres, les relations avec la FNAC, Darty et les autres, nous travaillons ensemble.\\
ML. : Pas de commerciaux ?\\
A.M. : Non, ce n’est pas utile. Nous verrons par la suite si nous n’y parvenons plus, mais pour le
moment, ça va.\\
ML. : Le marketing ?\\
A.M. : Un de nos ingénieurs, en fait, celui qui s’est occupé des relations avec l’entreprise de design a suffisamment la « fibre », et fait ça très bien. Ses collègues l’aident pour la création des brochures à partir des spécifications techniques. Vous savez, quand on a les meilleurs produits du marché, pas besoin de faire de publicité. Ça se sait vite ; le bouche à oreille, et maintenant les réseaux sociaux, et tout est dit. Les autres font le travail pour nous."}
\item \emph{"ML. : Bon ! Vous attaquez en justice la société ROTURO, qui commercialise également une montre
connectée. Que se passe-t-il dans ce petit monde ?\\
A.M. : Je serai très direct. Un de nos anciens employés a rejoint ce concurrent. Etant donnée sa position, il était au courant de nos développements, et nous ne croyons pas au hasard !"}
\end{itemize}
De plus il démontre une trop grande confiance en soi, il néglige son environnement :
\begin{itemize}
\item \emph{"A.M. : Bien évidemment, le plus abouti sur le plan technologique. C’est ce que cherchent les clients, n’est-ce pas ?\\
M.L. : Je n’ai pas à juger votre approche. Si elle fonctionne, c’est donc la bonne.\\
A.M. : C’est en tout cas la nôtre, et jusqu’à présent nous nous en portons plutôt bien.
M.L. : Et vous n’avez pas avant toute étude, cherché à savoir ce qui se passe dans le domaine, s’il existe un marché, … ?\\
A.M. : Pas la peine, compte tenu de notre savoir-faire, ..."}
\item \emph{"ML. : Vous pouvez être plus précis sur ces accusations ?\\
A.M. : Les éléments ont été donnés à notre avocat ! Nous sommes persuadés que la justice rendra une
décision en notre faveur."}
\item \emph{"Vous savez, quand on a les meilleurs produits du marché, pas besoin de faire de publicité"}
\end{itemize}

On peut ajouter à ça les membres de son équipe ne sont pas tous informés sur le caractère confidentiel des informations : 
\begin{itemize}
\item Nous ne savons pas trop encore ce qui va y être présenté, mais j’attends avec impatience. C’est fantastique toutes ces nouvelles possibilités offertes par les objets connectés.
\item Evidemment, non. Mais même si nous ne sommes pas très nombreux, ce n’est pas évident de sa
voir qui détient de l’information et si elle est à jour.
\end{itemize}

De plus parmi le dossier constitué pour l'étude indique qu'aucune traçabilité n'a été effectué dans l'entreprise : \emph{ Cela étant, il explique avoir toujours associé son logo à ses campagnes publicitaires, même s’il ne peut pas en rapporter la preuve aujourd’hui.}

\section{SWOT de la \textsc{sarl Firm}}
%--------------------------------------------------------------------------------------------------
%-----------------------------------------------SWOT-----------------------------------------------
%--------------------------------------------------------------------------------------------------
\begin{tabularx}{\linewidth}{|X|X|}
\hline
\hline
Forces & Faiblesses\\
\hline
\hline
Bonne communication transversale des informations techniques
& Aucune veille commerciale ou marketing \\
\hline
7 ans d'expérience dans le domaine, capital immatériel important
& Marketing fait par des ingénieurs potentiellement trop techniques\\
\hline
Mise en compétition des ingénieurs via des groupes de travail
& Aucune traçabilité des informations (Records information management)\\
\hline
Bonne ambiance
& Pas de service commercial au sens strict\\
\hline
Protection légale des produits (brevets)
& Veille technologique et scientifique limitée\\
\hline
& Veille concurrentielle inexistante\\
\hline
& Pas de rétention des employés\\
\hline
& Aucune gouvernance de l'information\\
\hline
& Aucune Veille image : mauvaise image présentée (locaux et employé qui s'en va)\\
\hline
& Aucune intelligence économique\\
\hline
& Non définition des ignorances\\
\hline
& Trop grande confiance en soi\\
\hline
\hline
Opportunités & Menaces\\
\hline
\hline
International Consumer Electronics Show (CES) 2016 en janvier
& Concurrent potentiel \textsc{Roturo}\\
\hline
Croissance du marché des objets connectés, vague du "quantified self"
& Procès pouvant entacher l'image de marque\\
\hline
Le procès de \textsc{Firm} permet de faire connaitre la marque
& Potentiel effet de mode des objets connectés et quantified self\\
\hline
Renommée dans le domaine
& La montre de \textsc{Roturo}: "connected clock"\\
\hline
& Parution d'information négative sur l'entreprise\\
\hline
\end{tabularx}

\section{SWOT de la \textsc{sas Roturo}}
\begin{tabularx}{\linewidth}{|X|X|}
\hline
\hline
Forces & Faiblesses\\
\hline
\hline
Focalisation sur le client
& Aucune vérification de la mise en application des politiques de gouvernance de l'information\\
\hline
Meilleures équipes car employés spécialisés dans leur domaine
& Sous-traitance des compétences clés \\
\hline
Politiques de la gouvernance d'information mises en place
& \\
\hline
Stratégie intéressante économiquement de part les sous-traitances 
& \\
\hline
Veille concurrentielle mise en place 
& \\
\hline
Bonne stratégie de communication, locaux modernes et bien placés 
& \\
\hline
Référentiel d'information (Records information management) opérationnel et audit possible 
& \\
\hline
\hline
Opportunités & Menaces\\
\hline
\hline
CES 2016 croissance du marché des objets connectés, vague du "quantified self"
& concurrent potentiel \textsc{Firm}\\
\hline
le procès de \textsc{Roturo} permet de faire connaitre la marque
& procès pouvant entacher l'image de marque\\
\hline
ouverture à l'international
& potentiel effet de mode des objets connectés et quantified self\\
\hline
& La montre de \textsc{Firm}: \textit{"connected watch"}\\
\hline
\end{tabularx}
%--------------------------------------------------------------------------------------------------
% ----------------------------------------END SWOT-----------------------------------------------
%--------------------------------------------------------------------------------------------------
\section{Les conseils}
\subsection{A mettre en place au sein de l'entreprise \textsc{Firm}}
\paragraph{Politique de gouvernance}
Pour pallier à certaines de ses faiblesses, il serait utile d'établir une politique de gouvernance de l'information. En répartissant les différentes informations dont dispose l'entreprise selon leur criticité et confidentialité dans une charte énonçant les types d'informations contenus dans les informations blanches, grises et noires, l'employé pourra savoir s'il a le droit de communiquer certaines informations. \\
Il est toutefois nécessaire d'accompagner la création de cette charte par des initiatives visant à contrôler le respect de cette charte. De plus, en établissant un référentiel d'informations et des compétences, l'employé en recherche d'information sera rapidement orienté vers la personne compétente ayant les informations les plus récentes et pertinentes sur le sujet. 
\paragraph{Politique de rétention}
Dans le domaine des employés, mettre en place, pour éviter le départ d'employés aux compétences et postes clefs de l'entreprise, une politique de rétention des employés visant à consolider les liens entre l'employé et la société limiterait les départs pour les entreprises concurrentes ou la fondation d'une nouvelle par un ancien employé.
\paragraph{Politique de veille}
La société \textsc{Firm} pourrait tirer profit de la mise en place d'une politique de veille à plusieurs niveaux. En effet, l'établissement d'une veille concurrentielle lui permettrait de rester au courant de ce qui existe ou est en développement dans son secteur d'activité. Ce type de veille pourrait notamment lui permettre de découvrir des produits concurrentiels avant leur sortie et d'adapter la stratégie commerciale de l'entreprise en conséquence. A cette veille peut s'ajouter une veille technologique, ce qui lui aurait notamment évité d'être pris au dépourvu lors de la sortie du produit de la société \textsc{Roturo}, lui permettant ainsi de commencer les procédures avant sa commercialisation. Pour permettre la bonne marche de ces veilles, il sera nécessaire de définir les ignorances de la société, c'est-à-dire ce qu'il est nécessaire de savoir ou ce qu'il est possible d'omettre dans les recherches d'informations, les concurrents existants ou potentiels à surveiller doivent notamment être définis.
\paragraph{Marketing et commercial}
Dans le domaine marketing et commercial, il pourrait être avantageux d'engager des employés dont la formation soit plus en adéquation avec le domaine. Un service marketing et commercial formé dans ces domaines agrandiraient le capital connaissance de l'entreprise tout en améliorant la communication de celle-ci en proposant notamment des idées qu'une personne formée à la technique n'aurait pas forcément eu. Ils pourront aussi remettre en adéquation les brochures et publicités de la société avec les attentes de la clientèle ciblés : des documentations techniques visant une clientèle de particulier ne semblent pas forcément adaptées. Enfin, cette équipe pourrait récupérer des informations directement auprès du consommateur pour connaître les nouvelles envies des utilisateurs et donc permettre des améliorations ou de nouveaux produits ayant un potentiel d'acceptation et de vente accru.
\paragraph{Campagne de communication}
Pour terminer, il pourrait être profitable d'effectuer une campagne de communication sur l'entreprise visant à améliorer l'image de l'entreprise. De récentes communications pourraient, à tort ou à raison, la qualifier d'arrogante et ainsi ternir cette image. Il en est de même pour les locaux, qui sont assimilables à une vitrine de l'entreprise, décrits péjorativement dans de récents articles. Un changement intérieur ou un déménagement pourrait être envisagé pour améliorer notre image et attirer plus facilement de futurs employés
\subsection{A l'encontre de l'entreprise \textsc{Roturo}}
La société \textsc{Firm} a tout intérêt à racheter le sous-traitant de la société \textsc{Roturo}. Cette action privera la \textsc{sas Roturo} de leurs compétences technique. Ainsi la société concurrente  serait dans l'incapacité de produire. Cet handicap pourrait entrainer leur faillite.\\
Le sous traitant peut ,de plus, être une solution de secours en cas de rachat ou autre problème avec la société \textsc{Clock}.
